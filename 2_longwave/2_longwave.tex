% Options for packages loaded elsewhere
\PassOptionsToPackage{unicode}{hyperref}
\PassOptionsToPackage{hyphens}{url}
%
\documentclass[
]{article}
\usepackage{amsmath,amssymb}
\usepackage{lmodern}
\usepackage{ifxetex,ifluatex}
\ifnum 0\ifxetex 1\fi\ifluatex 1\fi=0 % if pdftex
  \usepackage[T1]{fontenc}
  \usepackage[utf8]{inputenc}
  \usepackage{textcomp} % provide euro and other symbols
\else % if luatex or xetex
  \usepackage{unicode-math}
  \defaultfontfeatures{Scale=MatchLowercase}
  \defaultfontfeatures[\rmfamily]{Ligatures=TeX,Scale=1}
\fi
% Use upquote if available, for straight quotes in verbatim environments
\IfFileExists{upquote.sty}{\usepackage{upquote}}{}
\IfFileExists{microtype.sty}{% use microtype if available
  \usepackage[]{microtype}
  \UseMicrotypeSet[protrusion]{basicmath} % disable protrusion for tt fonts
}{}
\makeatletter
\@ifundefined{KOMAClassName}{% if non-KOMA class
  \IfFileExists{parskip.sty}{%
    \usepackage{parskip}
  }{% else
    \setlength{\parindent}{0pt}
    \setlength{\parskip}{6pt plus 2pt minus 1pt}}
}{% if KOMA class
  \KOMAoptions{parskip=half}}
\makeatother
\usepackage{xcolor}
\IfFileExists{xurl.sty}{\usepackage{xurl}}{} % add URL line breaks if available
\IfFileExists{bookmark.sty}{\usepackage{bookmark}}{\usepackage{hyperref}}
\hypersetup{
  pdftitle={2nd Protocol: Longwave},
  pdfauthor={Cheyenne Rueda and Simone Massaro},
  hidelinks,
  pdfcreator={LaTeX via pandoc}}
\urlstyle{same} % disable monospaced font for URLs
\usepackage[margin=1in]{geometry}
\usepackage{color}
\usepackage{fancyvrb}
\newcommand{\VerbBar}{|}
\newcommand{\VERB}{\Verb[commandchars=\\\{\}]}
\DefineVerbatimEnvironment{Highlighting}{Verbatim}{commandchars=\\\{\}}
% Add ',fontsize=\small' for more characters per line
\usepackage{framed}
\definecolor{shadecolor}{RGB}{248,248,248}
\newenvironment{Shaded}{\begin{snugshade}}{\end{snugshade}}
\newcommand{\AlertTok}[1]{\textcolor[rgb]{0.94,0.16,0.16}{#1}}
\newcommand{\AnnotationTok}[1]{\textcolor[rgb]{0.56,0.35,0.01}{\textbf{\textit{#1}}}}
\newcommand{\AttributeTok}[1]{\textcolor[rgb]{0.77,0.63,0.00}{#1}}
\newcommand{\BaseNTok}[1]{\textcolor[rgb]{0.00,0.00,0.81}{#1}}
\newcommand{\BuiltInTok}[1]{#1}
\newcommand{\CharTok}[1]{\textcolor[rgb]{0.31,0.60,0.02}{#1}}
\newcommand{\CommentTok}[1]{\textcolor[rgb]{0.56,0.35,0.01}{\textit{#1}}}
\newcommand{\CommentVarTok}[1]{\textcolor[rgb]{0.56,0.35,0.01}{\textbf{\textit{#1}}}}
\newcommand{\ConstantTok}[1]{\textcolor[rgb]{0.00,0.00,0.00}{#1}}
\newcommand{\ControlFlowTok}[1]{\textcolor[rgb]{0.13,0.29,0.53}{\textbf{#1}}}
\newcommand{\DataTypeTok}[1]{\textcolor[rgb]{0.13,0.29,0.53}{#1}}
\newcommand{\DecValTok}[1]{\textcolor[rgb]{0.00,0.00,0.81}{#1}}
\newcommand{\DocumentationTok}[1]{\textcolor[rgb]{0.56,0.35,0.01}{\textbf{\textit{#1}}}}
\newcommand{\ErrorTok}[1]{\textcolor[rgb]{0.64,0.00,0.00}{\textbf{#1}}}
\newcommand{\ExtensionTok}[1]{#1}
\newcommand{\FloatTok}[1]{\textcolor[rgb]{0.00,0.00,0.81}{#1}}
\newcommand{\FunctionTok}[1]{\textcolor[rgb]{0.00,0.00,0.00}{#1}}
\newcommand{\ImportTok}[1]{#1}
\newcommand{\InformationTok}[1]{\textcolor[rgb]{0.56,0.35,0.01}{\textbf{\textit{#1}}}}
\newcommand{\KeywordTok}[1]{\textcolor[rgb]{0.13,0.29,0.53}{\textbf{#1}}}
\newcommand{\NormalTok}[1]{#1}
\newcommand{\OperatorTok}[1]{\textcolor[rgb]{0.81,0.36,0.00}{\textbf{#1}}}
\newcommand{\OtherTok}[1]{\textcolor[rgb]{0.56,0.35,0.01}{#1}}
\newcommand{\PreprocessorTok}[1]{\textcolor[rgb]{0.56,0.35,0.01}{\textit{#1}}}
\newcommand{\RegionMarkerTok}[1]{#1}
\newcommand{\SpecialCharTok}[1]{\textcolor[rgb]{0.00,0.00,0.00}{#1}}
\newcommand{\SpecialStringTok}[1]{\textcolor[rgb]{0.31,0.60,0.02}{#1}}
\newcommand{\StringTok}[1]{\textcolor[rgb]{0.31,0.60,0.02}{#1}}
\newcommand{\VariableTok}[1]{\textcolor[rgb]{0.00,0.00,0.00}{#1}}
\newcommand{\VerbatimStringTok}[1]{\textcolor[rgb]{0.31,0.60,0.02}{#1}}
\newcommand{\WarningTok}[1]{\textcolor[rgb]{0.56,0.35,0.01}{\textbf{\textit{#1}}}}
\usepackage{graphicx}
\makeatletter
\def\maxwidth{\ifdim\Gin@nat@width>\linewidth\linewidth\else\Gin@nat@width\fi}
\def\maxheight{\ifdim\Gin@nat@height>\textheight\textheight\else\Gin@nat@height\fi}
\makeatother
% Scale images if necessary, so that they will not overflow the page
% margins by default, and it is still possible to overwrite the defaults
% using explicit options in \includegraphics[width, height, ...]{}
\setkeys{Gin}{width=\maxwidth,height=\maxheight,keepaspectratio}
% Set default figure placement to htbp
\makeatletter
\def\fps@figure{htbp}
\makeatother
\setlength{\emergencystretch}{3em} % prevent overfull lines
\providecommand{\tightlist}{%
  \setlength{\itemsep}{0pt}\setlength{\parskip}{0pt}}
\setcounter{secnumdepth}{5}
\usepackage{fancyhdr}
\pagestyle{fancy}

\fancyhead[R]{Experimental bioclimatology}
\fancyfoot[L]{Cheyenne Rueda \& Simone Massaro}
\fancyfoot[R]{\thepage}
\fancyfoot[C]{}
\fancyhead[L]{Longwave radiation}
\ifluatex
  \usepackage{selnolig}  % disable illegal ligatures
\fi

\title{2nd Protocol: Longwave}
\author{Cheyenne Rueda and Simone Massaro}
\date{29 mayo 2021}

\begin{document}
\maketitle

{
\setcounter{tocdepth}{2}
\tableofcontents
}
\newpage

\hypertarget{motivation}{%
\section{Motivation}\label{motivation}}

All bodies with a temperature above the absolute zero emit
electromagnetic radiation. The wavelength and intensity of this
radiation depends on the temperature of the body. In the temperature
range of the earth surface the emitted radiation has a wavelength
between 3 and 100 mm, and it is defined as longwave.

Emitting longwave radiation is the only way for the plant to cool itself
down, making it a crucial aspect in maintaining a steady climate. In
fact climate change is caused by green house emissions, which capture
part of the longwave radiation emitted by the planet surface and then
re-emit in back towards the Earth. This results in a bigger amount of
radiation reaching the earth surface, hence increasing its temperature
\footnote{J. E. Harries, The greenhouse Earth: A view from space, 1996
  \url{https://doi.org/10.1002/qj.49712253202}}.

Contrary to shortwave radiation, that is zero at night, longwave is a
constantly present in terrestrial ecosystem. Especially during night the
longwave radiation can have a substantial impact on the surface
temperature.

\hypertarget{background}{%
\section{Background}\label{background}}

Longwave radiation has an important impact on the research and
measurement of the global energy balance. Longwave radiation implements
big changes energy budget when modeling land surface. The measurement of
longwave radiation is more complicated in comparison with shortwave
radiation. This fact is due to the variables that are measured to
estimate shortwave radiation such as vapor pressure and air temperature.
Thus, longwave radiation estimation will differ according to the
atmospheric emissivity and temperature \footnote{Sridhar, V., \&
  Elliott, R. L. (2002). On the development of a simple downwelling
  longwave radiation scheme. Agricultural and Forest Meteorology,
  112(3-4), 237-243.}.

The changes produced in the radiation balance of the atmosphere are
induced by global warming as an adaptation to this process. These
changes are mainly caused by the variation in the level of water vapor
present in atmosphere. This is regularized with the absorbed solar
radiation and the emitted longwave radiation coming from atmosphere to
the surface of earth \footnote{Stephens, G. L., Wild, M., Stackhouse Jr,
  P. W., L'Ecuyer, T., Kato, S., \& Henderson, D. S. (2012). The global
  character of the flux of downward longwave radiation. \emph{Journal of
  Climate}, \emph{25}(7), 2329-2340.}.

In the paper written by Sridhar \& Elliott. (2002), some equations are
explained for the calculation of incoming longwave radiation.

Through the use of infrared sensors the calculation of outgoing
calculation can also be measured. Areas situated close to the ecuador
line will present higher amount of outgoing longwave radiation but also
higher temperature rates.

Longwave radiation is emitted by all bodies on the Earth, with the total
intensity depending on the temperature as described by the
\emph{Steffan-Boltzman law}.

\[E = \varepsilon \sigma T^4\] Where:

\begin{itemize}
\tightlist
\item
  \(E\) is the radiation intensity in \(W/m^2\)
\item
  \(\varepsilon\) an adimensional coefficient that represents the
  emissivity of the body. This depends on the material, a perfect black
  body has a \(\varepsilon\) of \(1\), while other materials have a
  lower emissivity
\item
  \(\sigma\) is the Steffan-Boltzman constant
  \(5.67 \times 10^{-8} W m^{-2} K^{-4}\)
\item
  \(T\) is the body temperature in \(K\)
\end{itemize}

As mentioned previously , green house gases emission are constantly
increasing. This is causing a higher levels of net temperature, process
induce by human known as global warming.

Other part that constitutes longwave radiation are the following terms:

\begin{itemize}
\item
  \emph{Atmospheric longwave radiation:} the emitted flux coming from
  clouds and water vapor particles present at atmosphere.
\item
  \emph{Surface longwave radiation:} (E), the reflected radiant flux
  from the planet's surface.
\end{itemize}

\[E = \varepsilon \sigma T^4\]

\begin{itemize}
\tightlist
\item
  \emph{Reflected longwave radiation:} (\(R_{lw}\)) is the reflected
  radiation occurring when the heat flux heats the surface.
\end{itemize}

\[R_{lw} = (1-\varepsilon)*A\]

\begin{itemize}
\tightlist
\item
  \emph{Longwace radiation balance:}
\end{itemize}

\[Q_{lw} =  A-E-R_{lw}\]

The principle followed to calculate the net radiation balance of the
planet is based on the sum of \(Q_{sw}\) and \(Q_{lw}\). Thus;

\[Qn = Qsw + Qlw\]
\[Qn=(1-rSW) G + \varepsilon \sigma *T_{atm}^4 - \varepsilon \sigma *T_s^4\]

Net radiation \((Rn)\) is defined by Bonan (2019), as the solar and
longwave radiation absorbed by the land surface. Another equation used
to calculate the land surface energy balance is known as:

\[Rn = (S_{incoming} - S_{outgoing}+(L_{incoming}) - L_{outgoing})\]
Also, emissivity can be defined as (1-\(\epsilon\)) and thus, the
fraction of longwave radiation that is reaching the earth surface and
being bounced back, can be explained (Bonan,2019). \footnote{Bonan, G.
  (2019). \emph{Climate change and terrestrial ecosystem modeling}.
  Cambridge University Press.}.

\hypertarget{sensors-and-measuring-principle}{%
\section{Sensors and measuring
principle}\label{sensors-and-measuring-principle}}

Longwave radiation is an important flux to understand how climatology
works in earth and understand processes such as global. Thus, some
instruments have been developed as well to measure longwave radiation.

The pyrgeometer measures this type of radiation from the atmosphere and
the Earth's surface. This device is based on a sensitive range with
infrared radiation between of 4.5 and 40 micrometer. It includes a
thermal detector (64 thermocouple thermophile) based on giving a voltage
output that then is needed to convert into radiative flux of energy
units (W m-2). The way it works is basic, calculating the difference
between the longwave radiation coming from the atmosphere and the
longwave produced by the sensor. This is represented by the next
equation:

\[ L_A= L_{net} + \sigma T_{sensor}^4 \]

Another device that allows to measure longwave radiation is the
pyrometer. The pyrometer measures longwave radiation and converts it
into a temperature based on the Stefan Boltzmann principle. Besides, it
exists other types of sensor based on the measurement of net radiation.
One of them is known as CNR4, based on 4 components (G, Rsw, A, E) which
gives an output value for each of these. The other instrument that can
be used for the same application is NR-Lite2. It is based on a
thermophile that measures net radiation (G+A) in the upper side, whether
in the other side it is measuring the net outgoing radiation (Rsw+E).
These last two are similar instruments, but the last one (NR-Lite2) only
gives one value as output in shape of voltage which is proportional to
the Rn or net radiation.

\hypertarget{analysis}{%
\section{Analysis}\label{analysis}}

\begin{Shaded}
\begin{Highlighting}[]
\NormalTok{rad }\OtherTok{\textless{}{-}} \FunctionTok{read\_csv}\NormalTok{(}\StringTok{"../Data\_lectures/2\_Longwave\_radiation/LW\_SW\_TSoil\_BotGarten.csv"}\NormalTok{)}
\FunctionTok{names}\NormalTok{(rad) }\OtherTok{\textless{}{-}} \FunctionTok{c}\NormalTok{(}\StringTok{"datetime"}\NormalTok{, }\StringTok{"t\_sens"}\NormalTok{, }\StringTok{"sw\_in"}\NormalTok{, }\StringTok{"sw\_out"}\NormalTok{, }\StringTok{"lw\_in\_sens"}\NormalTok{, }\StringTok{"lw\_out\_sens"}\NormalTok{, }\StringTok{"t\_soil"}\NormalTok{)}
\end{Highlighting}
\end{Shaded}

\begin{Shaded}
\begin{Highlighting}[]
\CommentTok{\# Utlitity funcs}
\NormalTok{sigma }\OtherTok{\textless{}{-}} \FloatTok{5.67e{-}8} 

\NormalTok{lw2temp }\OtherTok{\textless{}{-}} \ControlFlowTok{function}\NormalTok{(lw) (lw}\SpecialCharTok{/}\NormalTok{ sigma)}\SpecialCharTok{\^{}}\NormalTok{(}\DecValTok{1}\SpecialCharTok{/}\DecValTok{4}\NormalTok{)}
\NormalTok{temp2lw }\OtherTok{\textless{}{-}} \ControlFlowTok{function}\NormalTok{(temp)  }\FunctionTok{return}\NormalTok{ (sigma }\SpecialCharTok{*}\NormalTok{ temp}\SpecialCharTok{\^{}}\DecValTok{4}\NormalTok{)}

\NormalTok{c2k }\OtherTok{\textless{}{-}} \ControlFlowTok{function}\NormalTok{(c) c }\SpecialCharTok{+} \FloatTok{273.15}
\NormalTok{k2c }\OtherTok{\textless{}{-}} \ControlFlowTok{function}\NormalTok{(k) k }\SpecialCharTok{{-}} \FloatTok{273.15}
\end{Highlighting}
\end{Shaded}

\begin{Shaded}
\begin{Highlighting}[]
\CommentTok{\#calculate from input data the real lw and the soil/surface temperature}
\NormalTok{rad }\OtherTok{\textless{}{-}}\NormalTok{ rad }\SpecialCharTok{\%\textgreater{}\%}
  \FunctionTok{drop\_na}\NormalTok{() }\SpecialCharTok{\%\textgreater{}\%}
  \FunctionTok{mutate}\NormalTok{(}
    \AttributeTok{lw\_sens =} \FunctionTok{temp2lw}\NormalTok{(}\FunctionTok{c2k}\NormalTok{(t\_sens)),}
    \AttributeTok{lw\_in =}\NormalTok{ lw\_in\_sens }\SpecialCharTok{+}\NormalTok{ lw\_sens,}
    \AttributeTok{lw\_out =}\NormalTok{ lw\_out\_sens }\SpecialCharTok{+}\NormalTok{ lw\_sens,}
    \AttributeTok{t\_sky =} \FunctionTok{lw2temp}\NormalTok{(lw\_in) }\SpecialCharTok{\%\textgreater{}\%}\NormalTok{ k2c,}
    \AttributeTok{t\_surface =} \FunctionTok{lw2temp}\NormalTok{(lw\_out) }\SpecialCharTok{\%\textgreater{}\%}\NormalTok{ k2c,}
    \AttributeTok{net\_rad =}\NormalTok{ lw\_in }\SpecialCharTok{{-}}\NormalTok{ lw\_out }\SpecialCharTok{+}\NormalTok{ sw\_in }\SpecialCharTok{{-}}\NormalTok{ sw\_out,}
    \AttributeTok{net\_sw =}\NormalTok{ sw\_in }\SpecialCharTok{{-}}\NormalTok{ sw\_out,}
    \AttributeTok{net\_lw =}\NormalTok{ lw\_in }\SpecialCharTok{{-}}\NormalTok{ lw\_out}
\NormalTok{  )}
\end{Highlighting}
\end{Shaded}

\begin{Shaded}
\begin{Highlighting}[]
\CommentTok{\# for making aggregation easier we are going to consider data only for one calendar year}
\NormalTok{rad }\OtherTok{\textless{}{-}}\NormalTok{ rad }\SpecialCharTok{\%\textgreater{}\%}
  \FunctionTok{filter}\NormalTok{(datetime }\SpecialCharTok{\textless{}} \FunctionTok{as\_datetime}\NormalTok{(}\StringTok{"2020{-}12{-}31"}\NormalTok{))}
\end{Highlighting}
\end{Shaded}

\begin{Shaded}
\begin{Highlighting}[]
\CommentTok{\# weekly average data}
\NormalTok{rad\_w }\OtherTok{\textless{}{-}}\NormalTok{ rad }\SpecialCharTok{\%\textgreater{}\%} 
  \FunctionTok{as\_tsibble}\NormalTok{(}\AttributeTok{index =}\NormalTok{ datetime) }\SpecialCharTok{\%\textgreater{}\%}
  \FunctionTok{index\_by}\NormalTok{(}\AttributeTok{week =} \SpecialCharTok{\textasciitilde{}} \FunctionTok{yearweek}\NormalTok{(.)) }\SpecialCharTok{\%\textgreater{}\%}
  \FunctionTok{summarise\_all}\NormalTok{(mean, }\AttributeTok{na.rm =} \ConstantTok{TRUE}\NormalTok{)}
\end{Highlighting}
\end{Shaded}

\begin{Shaded}
\begin{Highlighting}[]
\CommentTok{\# daily average data}
\NormalTok{rad\_d }\OtherTok{\textless{}{-}}\NormalTok{ rad }\SpecialCharTok{\%\textgreater{}\%}
  \FunctionTok{mutate}\NormalTok{(}\AttributeTok{yday =} \FunctionTok{yday}\NormalTok{(datetime)) }\SpecialCharTok{\%\textgreater{}\%}
  \FunctionTok{group\_by}\NormalTok{(yday) }\SpecialCharTok{\%\textgreater{}\%}
  \FunctionTok{summarize\_all}\NormalTok{(mean, }\AttributeTok{na.rm =} \ConstantTok{TRUE}\NormalTok{)}
\end{Highlighting}
\end{Shaded}

\newpage

\hypertarget{surface-and-sky-temperature}{%
\subsection{Surface and Sky
temperature}\label{surface-and-sky-temperature}}

\emph{Derive the sky and surface temperature from the longwave radiation
components. How do they differ and why? Discuss! During which periods of
the year sky and surface temperature differ the most and the less?}

The two temperature are analyzed at different timescales

\begin{Shaded}
\begin{Highlighting}[]
\NormalTok{rad\_w }\SpecialCharTok{\%\textgreater{}\%}
  \FunctionTok{gather}\NormalTok{(}\AttributeTok{key=}\StringTok{"type"}\NormalTok{, }\AttributeTok{value=}\StringTok{"temp"}\NormalTok{, t\_sky, t\_surface) }\SpecialCharTok{\%\textgreater{}\%}
\FunctionTok{ggplot}\NormalTok{(}\FunctionTok{aes}\NormalTok{(}\AttributeTok{x=}\NormalTok{datetime, }\AttributeTok{y=}\NormalTok{temp, }\AttributeTok{colour=}\NormalTok{type)) }\SpecialCharTok{+}
  \FunctionTok{geom\_line}\NormalTok{() }\SpecialCharTok{+}
  \FunctionTok{labs}\NormalTok{(}\AttributeTok{caption=}\StringTok{"Weekly average"}\NormalTok{, }\AttributeTok{y=}\StringTok{"Temperature (°C)"}\NormalTok{, }\AttributeTok{x=}\StringTok{"Time"}\NormalTok{,}
       \AttributeTok{title=}\StringTok{"Sky and Surface temperatures over the year"}\NormalTok{)}
\end{Highlighting}
\end{Shaded}

\includegraphics{2_longwave_files/figure-latex/unnamed-chunk-8-1.pdf}
\newpage

\begin{Shaded}
\begin{Highlighting}[]
\NormalTok{rad\_d }\SpecialCharTok{\%\textgreater{}\%}
  \FunctionTok{gather}\NormalTok{(}\AttributeTok{key=}\StringTok{"type"}\NormalTok{, }\AttributeTok{value=}\StringTok{"temp"}\NormalTok{, t\_sky, t\_surface) }\SpecialCharTok{\%\textgreater{}\%}
\FunctionTok{ggplot}\NormalTok{(}\FunctionTok{aes}\NormalTok{(}\AttributeTok{x=}\NormalTok{datetime, }\AttributeTok{y=}\NormalTok{temp, }\AttributeTok{colour=}\NormalTok{type)) }\SpecialCharTok{+}
  \FunctionTok{geom\_line}\NormalTok{()  }\SpecialCharTok{+}
  \FunctionTok{labs}\NormalTok{(}\AttributeTok{caption=}\StringTok{"Daily average"}\NormalTok{, }\AttributeTok{y=}\StringTok{"Temperature (°C)"}\NormalTok{, }\AttributeTok{x=}\StringTok{"Time"}\NormalTok{,}
       \AttributeTok{title=}\StringTok{"Sky and Surface temperatures over the year"}\NormalTok{)}
\end{Highlighting}
\end{Shaded}

\includegraphics{2_longwave_files/figure-latex/unnamed-chunk-9-1.pdf}

The sky temperatures is always lower than the surface one. The surface
temperature ranges from -2 C to 25 C, while the sky temperature has a
bigger range from -23 C to 17 C. The temperature of the sky mainly
depends on the cloud cover and the temperature of the air. \newpage

\begin{Shaded}
\begin{Highlighting}[]
\NormalTok{rad }\SpecialCharTok{\%\textgreater{}\%}
  \FunctionTok{filter}\NormalTok{( }\FunctionTok{month}\NormalTok{(datetime) }\SpecialCharTok{==} \DecValTok{7}\NormalTok{ ) }\SpecialCharTok{\%\textgreater{}\%}
  \FunctionTok{gather}\NormalTok{(}\AttributeTok{key=}\StringTok{"type"}\NormalTok{, }\AttributeTok{value=}\StringTok{"temp"}\NormalTok{, t\_sky, t\_surface) }\SpecialCharTok{\%\textgreater{}\%}
\FunctionTok{ggplot}\NormalTok{(}\FunctionTok{aes}\NormalTok{(}\AttributeTok{x=}\NormalTok{datetime, }\AttributeTok{y=}\NormalTok{temp, }\AttributeTok{colour=}\NormalTok{type)) }\SpecialCharTok{+}
  \FunctionTok{geom\_line}\NormalTok{()  }\SpecialCharTok{+}
  \FunctionTok{labs}\NormalTok{(}\AttributeTok{caption=}\StringTok{"Month of July (10 mins data)"}\NormalTok{, }\AttributeTok{y=}\StringTok{"Temperature (°C)"}\NormalTok{, }\AttributeTok{x=}\StringTok{"Time"}\NormalTok{,}
       \AttributeTok{title=}\StringTok{"Sky and Surface temperatures over one month"}\NormalTok{)}
\end{Highlighting}
\end{Shaded}

\includegraphics{2_longwave_files/figure-latex/unnamed-chunk-10-1.pdf}
\newpage

\begin{Shaded}
\begin{Highlighting}[]
\FunctionTok{filter}\NormalTok{(rad, }\FunctionTok{between}\NormalTok{(datetime, }\FunctionTok{as\_datetime}\NormalTok{(}\StringTok{"2020{-}07{-}03"}\NormalTok{), }\FunctionTok{as\_datetime}\NormalTok{(}\StringTok{"2020{-}07{-}12"}\NormalTok{))) }\SpecialCharTok{\%\textgreater{}\%}
  \FunctionTok{gather}\NormalTok{(}\AttributeTok{key=}\StringTok{"type"}\NormalTok{, }\AttributeTok{value=}\StringTok{"temp"}\NormalTok{, t\_sky, t\_surface) }\SpecialCharTok{\%\textgreater{}\%}
\FunctionTok{ggplot}\NormalTok{(}\FunctionTok{aes}\NormalTok{(}\AttributeTok{x=}\NormalTok{datetime, }\AttributeTok{y=}\NormalTok{temp, }\AttributeTok{colour=}\NormalTok{type)) }\SpecialCharTok{+}
  \FunctionTok{geom\_line}\NormalTok{()  }\SpecialCharTok{+}
  \FunctionTok{labs}\NormalTok{(}\AttributeTok{caption=}\StringTok{"3{-}12 July (10 mins data)"}\NormalTok{, }\AttributeTok{y=}\StringTok{"Temperature (°C)"}\NormalTok{, }\AttributeTok{x=}\StringTok{"Time"}\NormalTok{,}
       \AttributeTok{title=}\StringTok{"Sky and Surface temperatures over one week"}\NormalTok{)}
\end{Highlighting}
\end{Shaded}

\includegraphics{2_longwave_files/figure-latex/unnamed-chunk-11-1.pdf}

The surface temperature has a clear day cycle, while the sky temperature
has little or no day pattern. Moreover, it can be clearly seen how
cloudy days (eg. 9th of July) there is a high sky temperature, but a low
surface temperature. Conversely, on sunny days (eg. 7th of July) the
surface temperature is higher, but the sky temperature is low.

\newpage

\hypertarget{net-radiation}{%
\subsection{Net radiation}\label{net-radiation}}

\emph{Calculate the net radiation over the meadow in the forest
botanical garden and plot the four components. How do the four
components change over the season and why? Which unexpected results you
found? Discuss!}

\begin{Shaded}
\begin{Highlighting}[]
\NormalTok{rad\_w }\SpecialCharTok{\%\textgreater{}\%}
  \FunctionTok{gather}\NormalTok{(}\AttributeTok{key=}\StringTok{"type"}\NormalTok{, }\AttributeTok{value=}\StringTok{"radiation"}\NormalTok{, net\_rad, lw\_in, lw\_out, sw\_in, sw\_out,}
         \AttributeTok{factor\_key =}\NormalTok{ T) }\SpecialCharTok{\%\textgreater{}\%}
\FunctionTok{ggplot}\NormalTok{() }\SpecialCharTok{+}
  \FunctionTok{geom\_line}\NormalTok{(}\FunctionTok{aes}\NormalTok{(}\AttributeTok{x=}\NormalTok{datetime, }\AttributeTok{y=}\NormalTok{radiation, }\AttributeTok{colour=}\NormalTok{type)) }\SpecialCharTok{+}
  \FunctionTok{geom\_line}\NormalTok{(}\FunctionTok{aes}\NormalTok{(}\AttributeTok{x=}\NormalTok{datetime, }\AttributeTok{y=}\NormalTok{net\_rad), }\AttributeTok{data=}\NormalTok{rad\_w, }\AttributeTok{size=}\FloatTok{1.2}\NormalTok{,}
            \AttributeTok{colour=}\FunctionTok{hue\_pal}\NormalTok{()(}\DecValTok{1}\NormalTok{)) }\SpecialCharTok{+}
  \FunctionTok{labs}\NormalTok{(}\AttributeTok{title=}\StringTok{"Net radiation with components over "}\NormalTok{, }\AttributeTok{y=}\StringTok{"Radiation (W m{-}2)"}\NormalTok{,}
       \AttributeTok{x=}\StringTok{"Time"}\NormalTok{, }\AttributeTok{caption =} \StringTok{"Weekly average"}\NormalTok{, }\AttributeTok{colour=}\StringTok{"Radiation"}\NormalTok{)}
\end{Highlighting}
\end{Shaded}

\includegraphics{2_longwave_files/figure-latex/unnamed-chunk-12-1.pdf}

The net radiation has a yearly cycle. During the summer it has a
relatively constant value at around \(100 W/m^2\), then it decrease and
reach slightly negative values in January. The biggest driver of this
yearly cycle is the incoming shortwave radiation, which during summer is
much higher than in winter. The radiation from the sun has smooth
variations, while the variation on the incoming shortwave during the
summer can be explained by the different amount of cloud cover. You
would expect a clearer peak of the shortwave radiation during the
summer, Moreover the net radiation has an high peak in mid late
September. This behavior can probably be explained by different amount
of cloud cover.

The outgoing shortwave is the component with the smallest absolute
value, it also has a yearly cycle being virtually zero in January but
quickly reaching the max value during the spring and then remaining
quite flat. Regarding the longwave the outgoing radiation is always
bigger than the incoming, due to the higher temperature of the surface
compared to the sky. The longwave components have a much smaller change
during the year.

There is a notable low peak of incoming longwave in the last week of
march, that is probably explained by clear skies but still low air
temperature.

\newpage

\hypertarget{change-emissity-in-the-sensor}{%
\subsection{Change emissity in the
sensor}\label{change-emissity-in-the-sensor}}

In the field activity we tried to measure the temperature of the surface
by using different emissivity settings in the sensor and see how that
could influence the readings. However, there have been some issues with
the sensor, so the data has been generated using the formula from the
theory

In this virtual experiment the real temperature is set to \(19 °C\) and
the emissivity is changed, resulting in different temperature estimates.

\begin{Shaded}
\begin{Highlighting}[]
\NormalTok{t\_0 }\OtherTok{\textless{}{-}} \DecValTok{19} \CommentTok{\# temperature with emissivity 1}
\NormalTok{rad\_0 }\OtherTok{\textless{}{-}} \FunctionTok{c2k}\NormalTok{(t\_0) }\SpecialCharTok{\%\textgreater{}\%}\NormalTok{ temp2lw }\CommentTok{\# connected radiation }

\NormalTok{temps }\OtherTok{\textless{}{-}} \FunctionTok{tibble}\NormalTok{(}
  \AttributeTok{em =} \FunctionTok{seq}\NormalTok{(}\DecValTok{1}\NormalTok{, .}\DecValTok{1}\NormalTok{, }\SpecialCharTok{{-}}\NormalTok{.}\DecValTok{05}\NormalTok{),}
  \AttributeTok{t =}\NormalTok{ (rad\_0 }\SpecialCharTok{/}\NormalTok{ (em }\SpecialCharTok{*}\NormalTok{ sigma))}\SpecialCharTok{\^{}}\NormalTok{(}\DecValTok{1}\SpecialCharTok{/}\DecValTok{4}\NormalTok{)  }\SpecialCharTok{\%\textgreater{}\%}\NormalTok{ k2c}
\NormalTok{) }
\end{Highlighting}
\end{Shaded}

\begin{Shaded}
\begin{Highlighting}[]
\FunctionTok{ggplot}\NormalTok{(temps, }\FunctionTok{aes}\NormalTok{(em, t)) }\SpecialCharTok{+}
  \FunctionTok{geom\_line}\NormalTok{() }\SpecialCharTok{+}
  \FunctionTok{scale\_x\_reverse}\NormalTok{() }\SpecialCharTok{+}
  \FunctionTok{labs}\NormalTok{(}\AttributeTok{x=}\StringTok{"Emissivity"}\NormalTok{, }\AttributeTok{y=}\StringTok{"Temperature"}\NormalTok{, }\AttributeTok{title=}\StringTok{"Different temperatures with different emissivity"}\NormalTok{)}
\end{Highlighting}
\end{Shaded}

\includegraphics{2_longwave_files/figure-latex/unnamed-chunk-14-1.pdf}

It can be seen that the emissivity has a big influence on the
temperature estimate. The emissivity of material can change drastically
from \(0.03\) for aluminum foil to \(0.97\) for ice. This clearly shoes
the importance of a correct estimation of the emissivity for temperature
measurements.

\end{document}
