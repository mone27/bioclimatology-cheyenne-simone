% Options for packages loaded elsewhere
\PassOptionsToPackage{unicode}{hyperref}
\PassOptionsToPackage{hyphens}{url}
%
\documentclass[
]{article}
\usepackage{amsmath,amssymb}
\usepackage{lmodern}
\usepackage{ifxetex,ifluatex}
\ifnum 0\ifxetex 1\fi\ifluatex 1\fi=0 % if pdftex
  \usepackage[T1]{fontenc}
  \usepackage[utf8]{inputenc}
  \usepackage{textcomp} % provide euro and other symbols
\else % if luatex or xetex
  \usepackage{unicode-math}
  \defaultfontfeatures{Scale=MatchLowercase}
  \defaultfontfeatures[\rmfamily]{Ligatures=TeX,Scale=1}
\fi
% Use upquote if available, for straight quotes in verbatim environments
\IfFileExists{upquote.sty}{\usepackage{upquote}}{}
\IfFileExists{microtype.sty}{% use microtype if available
  \usepackage[]{microtype}
  \UseMicrotypeSet[protrusion]{basicmath} % disable protrusion for tt fonts
}{}
\makeatletter
\@ifundefined{KOMAClassName}{% if non-KOMA class
  \IfFileExists{parskip.sty}{%
    \usepackage{parskip}
  }{% else
    \setlength{\parindent}{0pt}
    \setlength{\parskip}{6pt plus 2pt minus 1pt}}
}{% if KOMA class
  \KOMAoptions{parskip=half}}
\makeatother
\usepackage{xcolor}
\IfFileExists{xurl.sty}{\usepackage{xurl}}{} % add URL line breaks if available
\IfFileExists{bookmark.sty}{\usepackage{bookmark}}{\usepackage{hyperref}}
\hypersetup{
  pdftitle={5th Protocol: Precipitation},
  pdfauthor={Cheyenne Rueda and Simone Massaro},
  hidelinks,
  pdfcreator={LaTeX via pandoc}}
\urlstyle{same} % disable monospaced font for URLs
\usepackage[margin=1in]{geometry}
\usepackage{color}
\usepackage{fancyvrb}
\newcommand{\VerbBar}{|}
\newcommand{\VERB}{\Verb[commandchars=\\\{\}]}
\DefineVerbatimEnvironment{Highlighting}{Verbatim}{commandchars=\\\{\}}
% Add ',fontsize=\small' for more characters per line
\usepackage{framed}
\definecolor{shadecolor}{RGB}{248,248,248}
\newenvironment{Shaded}{\begin{snugshade}}{\end{snugshade}}
\newcommand{\AlertTok}[1]{\textcolor[rgb]{0.94,0.16,0.16}{#1}}
\newcommand{\AnnotationTok}[1]{\textcolor[rgb]{0.56,0.35,0.01}{\textbf{\textit{#1}}}}
\newcommand{\AttributeTok}[1]{\textcolor[rgb]{0.77,0.63,0.00}{#1}}
\newcommand{\BaseNTok}[1]{\textcolor[rgb]{0.00,0.00,0.81}{#1}}
\newcommand{\BuiltInTok}[1]{#1}
\newcommand{\CharTok}[1]{\textcolor[rgb]{0.31,0.60,0.02}{#1}}
\newcommand{\CommentTok}[1]{\textcolor[rgb]{0.56,0.35,0.01}{\textit{#1}}}
\newcommand{\CommentVarTok}[1]{\textcolor[rgb]{0.56,0.35,0.01}{\textbf{\textit{#1}}}}
\newcommand{\ConstantTok}[1]{\textcolor[rgb]{0.00,0.00,0.00}{#1}}
\newcommand{\ControlFlowTok}[1]{\textcolor[rgb]{0.13,0.29,0.53}{\textbf{#1}}}
\newcommand{\DataTypeTok}[1]{\textcolor[rgb]{0.13,0.29,0.53}{#1}}
\newcommand{\DecValTok}[1]{\textcolor[rgb]{0.00,0.00,0.81}{#1}}
\newcommand{\DocumentationTok}[1]{\textcolor[rgb]{0.56,0.35,0.01}{\textbf{\textit{#1}}}}
\newcommand{\ErrorTok}[1]{\textcolor[rgb]{0.64,0.00,0.00}{\textbf{#1}}}
\newcommand{\ExtensionTok}[1]{#1}
\newcommand{\FloatTok}[1]{\textcolor[rgb]{0.00,0.00,0.81}{#1}}
\newcommand{\FunctionTok}[1]{\textcolor[rgb]{0.00,0.00,0.00}{#1}}
\newcommand{\ImportTok}[1]{#1}
\newcommand{\InformationTok}[1]{\textcolor[rgb]{0.56,0.35,0.01}{\textbf{\textit{#1}}}}
\newcommand{\KeywordTok}[1]{\textcolor[rgb]{0.13,0.29,0.53}{\textbf{#1}}}
\newcommand{\NormalTok}[1]{#1}
\newcommand{\OperatorTok}[1]{\textcolor[rgb]{0.81,0.36,0.00}{\textbf{#1}}}
\newcommand{\OtherTok}[1]{\textcolor[rgb]{0.56,0.35,0.01}{#1}}
\newcommand{\PreprocessorTok}[1]{\textcolor[rgb]{0.56,0.35,0.01}{\textit{#1}}}
\newcommand{\RegionMarkerTok}[1]{#1}
\newcommand{\SpecialCharTok}[1]{\textcolor[rgb]{0.00,0.00,0.00}{#1}}
\newcommand{\SpecialStringTok}[1]{\textcolor[rgb]{0.31,0.60,0.02}{#1}}
\newcommand{\StringTok}[1]{\textcolor[rgb]{0.31,0.60,0.02}{#1}}
\newcommand{\VariableTok}[1]{\textcolor[rgb]{0.00,0.00,0.00}{#1}}
\newcommand{\VerbatimStringTok}[1]{\textcolor[rgb]{0.31,0.60,0.02}{#1}}
\newcommand{\WarningTok}[1]{\textcolor[rgb]{0.56,0.35,0.01}{\textbf{\textit{#1}}}}
\usepackage{graphicx}
\makeatletter
\def\maxwidth{\ifdim\Gin@nat@width>\linewidth\linewidth\else\Gin@nat@width\fi}
\def\maxheight{\ifdim\Gin@nat@height>\textheight\textheight\else\Gin@nat@height\fi}
\makeatother
% Scale images if necessary, so that they will not overflow the page
% margins by default, and it is still possible to overwrite the defaults
% using explicit options in \includegraphics[width, height, ...]{}
\setkeys{Gin}{width=\maxwidth,height=\maxheight,keepaspectratio}
% Set default figure placement to htbp
\makeatletter
\def\fps@figure{htbp}
\makeatother
\setlength{\emergencystretch}{3em} % prevent overfull lines
\providecommand{\tightlist}{%
  \setlength{\itemsep}{0pt}\setlength{\parskip}{0pt}}
\setcounter{secnumdepth}{-\maxdimen} % remove section numbering
\usepackage{fancyhdr}
\pagestyle{fancy}

\fancyhead[R]{Experimental bioclimatology}
\fancyfoot[L]{Cheyenne Rueda \& Simone Massaro}
\fancyfoot[R]{\thepage}
\fancyfoot[C]{}
\fancyhead[L]{Shortwave radiation}
\ifluatex
  \usepackage{selnolig}  % disable illegal ligatures
\fi

\title{5th Protocol: Precipitation}
\author{Cheyenne Rueda and Simone Massaro}
\date{31 mayo 2021}

\begin{document}
\maketitle

{
\setcounter{tocdepth}{2}
\tableofcontents
}
\hypertarget{analysis}{%
\subsection{Analysis}\label{analysis}}

\begin{Shaded}
\begin{Highlighting}[]
\NormalTok{prec }\OtherTok{\textless{}{-}} \FunctionTok{read\_csv}\NormalTok{(}\StringTok{"../Data\_lectures/5\_Precipitation/P\_4sites.csv"}\NormalTok{)}
\NormalTok{et }\OtherTok{\textless{}{-}} \FunctionTok{read\_csv}\NormalTok{(}\StringTok{"../Data\_lectures/5\_Precipitation/ET\_4sites.csv"}\NormalTok{)}
\NormalTok{sites }\OtherTok{\textless{}{-}} \FunctionTok{read\_csv}\NormalTok{(}\StringTok{"station\_data.csv"}\NormalTok{) }\SpecialCharTok{\%\textgreater{}\%}
  \FunctionTok{rename}\NormalTok{(}\AttributeTok{site=}\StringTok{\textasciigrave{}}\AttributeTok{Site{-}abb}\StringTok{\textasciigrave{}}\NormalTok{, }\AttributeTok{full\_name =}\NormalTok{ Site)}
\end{Highlighting}
\end{Shaded}

\begin{Shaded}
\begin{Highlighting}[]
\NormalTok{prec\_avg }\OtherTok{\textless{}{-}}\NormalTok{ prec }\SpecialCharTok{\%\textgreater{}\%}
  \FunctionTok{select}\NormalTok{(}\SpecialCharTok{{-}}\NormalTok{Date) }\SpecialCharTok{\%\textgreater{}\%}
  \FunctionTok{gather}\NormalTok{(}\StringTok{"site"}\NormalTok{, }\StringTok{"prec"}\NormalTok{) }\SpecialCharTok{\%\textgreater{}\%}
    \FunctionTok{group\_by}\NormalTok{(site) }\SpecialCharTok{\%\textgreater{}\%}
  \FunctionTok{summarise}\NormalTok{(}\AttributeTok{prec=}\FunctionTok{mean}\NormalTok{(prec))}

\NormalTok{et\_avg }\OtherTok{\textless{}{-}}\NormalTok{ et }\SpecialCharTok{\%\textgreater{}\%}
  \FunctionTok{select}\NormalTok{(}\SpecialCharTok{{-}}\NormalTok{Date) }\SpecialCharTok{\%\textgreater{}\%}
  \FunctionTok{gather}\NormalTok{(}\StringTok{"site"}\NormalTok{, }\StringTok{"et"}\NormalTok{) }\SpecialCharTok{\%\textgreater{}\%}
    \FunctionTok{group\_by}\NormalTok{(site) }\SpecialCharTok{\%\textgreater{}\%}
  \FunctionTok{summarise}\NormalTok{(}\AttributeTok{et=}\FunctionTok{mean}\NormalTok{(et))}

\NormalTok{sites\_avg }\OtherTok{\textless{}{-}}\NormalTok{ sites }\SpecialCharTok{\%\textgreater{}\%}
  \FunctionTok{inner\_join}\NormalTok{(et\_avg, }\AttributeTok{by=}\StringTok{"site"}\NormalTok{) }\SpecialCharTok{\%\textgreater{}\%}
  \FunctionTok{inner\_join}\NormalTok{(prec\_avg, }\AttributeTok{by=}\StringTok{"site"}\NormalTok{)}
\end{Highlighting}
\end{Shaded}

\hypertarget{et-and-precipitation-at-different-latitudes}{%
\subsection{ET and precipitation at different
latitudes}\label{et-and-precipitation-at-different-latitudes}}

\emph{Visualise the annual sums of evapotranspiration and precipitation
according to its geographic latitude (boxplot?). How and why does
precipitation and evapotranspiration varies? Which impact has the
underlying ecosystem?}

\begin{Shaded}
\begin{Highlighting}[]
\CommentTok{\# this t makes no sense as name }\AlertTok{TODO}\CommentTok{ chanve this}
\NormalTok{prec\_t }\OtherTok{\textless{}{-}}\NormalTok{ prec }\SpecialCharTok{\%\textgreater{}\%}
  \FunctionTok{gather}\NormalTok{(}\StringTok{"site"}\NormalTok{, }\StringTok{"prec"}\NormalTok{, }\SpecialCharTok{{-}}\NormalTok{Date)}

\NormalTok{et\_t }\OtherTok{\textless{}{-}}\NormalTok{ et }\SpecialCharTok{\%\textgreater{}\%}
  \FunctionTok{gather}\NormalTok{(}\StringTok{"site"}\NormalTok{, }\StringTok{"et"}\NormalTok{, }\SpecialCharTok{{-}}\NormalTok{Date) }

\NormalTok{prec\_et }\OtherTok{\textless{}{-}} \FunctionTok{left\_join}\NormalTok{(prec\_t, et\_t, }\AttributeTok{by=}\FunctionTok{c}\NormalTok{(}\StringTok{"Date"}\NormalTok{, }\StringTok{"site"}\NormalTok{))}

\NormalTok{sites }\OtherTok{\textless{}{-}}\NormalTok{ sites }\SpecialCharTok{\%\textgreater{}\%}
  \FunctionTok{inner\_join}\NormalTok{(prec\_et, }\AttributeTok{by=}\StringTok{"site"}\NormalTok{) }
\end{Highlighting}
\end{Shaded}

\begin{Shaded}
\begin{Highlighting}[]
\NormalTok{sites}
\end{Highlighting}
\end{Shaded}

\begin{verbatim}
## # A tibble: 24 x 11
##    site   full_name   Lat  Long Height Veg_type Mean_airT Mean_precipitation
##    <chr>  <chr>     <dbl> <dbl>  <dbl> <chr>        <dbl>              <dbl>
##  1 DE_Hai Hainich    51.1  10.5    430 DBF            8.3                720
##  2 DE_Hai Hainich    51.1  10.5    430 DBF            8.3                720
##  3 DE_Hai Hainich    51.1  10.5    430 DBF            8.3                720
##  4 DE_Hai Hainich    51.1  10.5    430 DBF            8.3                720
##  5 DE_Hai Hainich    51.1  10.5    430 DBF            8.3                720
##  6 DE_Hai Hainich    51.1  10.5    430 DBF            8.3                720
##  7 FI_Hyy Hyytiala   61.8  24.3    181 ENF            3.8                709
##  8 FI_Hyy Hyytiala   61.8  24.3    181 ENF            3.8                709
##  9 FI_Hyy Hyytiala   61.8  24.3    181 ENF            3.8                709
## 10 FI_Hyy Hyytiala   61.8  24.3    181 ENF            3.8                709
## # ... with 14 more rows, and 3 more variables: Date <date>, prec <dbl>,
## #   et <dbl>
\end{verbatim}

\begin{Shaded}
\begin{Highlighting}[]
\NormalTok{(prec\_box }\OtherTok{\textless{}{-}}  \FunctionTok{ggplot}\NormalTok{(sites, }\FunctionTok{aes}\NormalTok{(Lat, prec, }\AttributeTok{colour=}\NormalTok{site)) }\SpecialCharTok{+}
  \FunctionTok{geom\_boxplot}\NormalTok{() }\SpecialCharTok{+}
   \FunctionTok{labs}\NormalTok{(}\AttributeTok{title=}\StringTok{"Cumlative precipitation at different latitudes"}\NormalTok{,}
        \AttributeTok{y=}\StringTok{"Total yearly precipitation (mm)"}\NormalTok{,}
        \AttributeTok{x=}\StringTok{"Latitude (deg)"}\NormalTok{))}
\end{Highlighting}
\end{Shaded}

\includegraphics{5_precipitation_files/figure-latex/unnamed-chunk-6-1.pdf}

\begin{Shaded}
\begin{Highlighting}[]
\NormalTok{(et\_box }\OtherTok{\textless{}{-}} \FunctionTok{ggplot}\NormalTok{(sites, }\FunctionTok{aes}\NormalTok{(Lat, et, }\AttributeTok{colour=}\NormalTok{site)) }\SpecialCharTok{+}
  \FunctionTok{geom\_boxplot}\NormalTok{() }\SpecialCharTok{+}
    \FunctionTok{labs}\NormalTok{(}\AttributeTok{title=}\StringTok{"Cumlative evapotranspiration at different latitudes"}\NormalTok{,}
        \AttributeTok{y=}\StringTok{"Total yearly evapotranspiratio (mm)"}\NormalTok{,}
        \AttributeTok{x=}\StringTok{"Latitude (deg)"}\NormalTok{))}
\end{Highlighting}
\end{Shaded}

\includegraphics{5_precipitation_files/figure-latex/unnamed-chunk-7-1.pdf}
\#\# Evapotranspiration index

\emph{Precipitation is used as a driver/source for evapotranspiration.
The ratio between evapotranspiration and precipitation
(evapotranspiration index = ET\_idx) is a very useful indicator how
precipitation is used by plants during photosynthesis. Calculate the
ET\_idx for all sites and visualise it. What surprised you and how do
the ET\_idx differs across ecosystems? Discuss.}

\begin{Shaded}
\begin{Highlighting}[]
\NormalTok{sites}\SpecialCharTok{$}\NormalTok{et\_idx }\OtherTok{\textless{}{-}}\NormalTok{ sites}\SpecialCharTok{$}\NormalTok{et}\SpecialCharTok{/}\NormalTok{sites}\SpecialCharTok{$}\NormalTok{prec}

\FunctionTok{ggplot}\NormalTok{(sites,}
       \FunctionTok{aes}\NormalTok{(site, et\_idx, }\AttributeTok{colour =}\NormalTok{ site))}\SpecialCharTok{+}
         \FunctionTok{geom\_boxplot}\NormalTok{()}\SpecialCharTok{+}
         \FunctionTok{labs}\NormalTok{(}\AttributeTok{title=}\StringTok{"Evapotranspiration index"}\NormalTok{, }\AttributeTok{x=}\StringTok{"Different site"}\NormalTok{, }\AttributeTok{y=}\StringTok{"Evapotranspiration index"}\NormalTok{)}
\end{Highlighting}
\end{Shaded}

\includegraphics{5_precipitation_files/figure-latex/unnamed-chunk-8-1.pdf}

As we can see,evapotranspiration index is the fraction between what is
intercepted by cannopy and what is being lose by soil hydrology
processes. Other variables such as wind speed, surface humidity and
solar radiation are not being taken into account in the calculation of
this index, which probably are the ones causing differences on what it
could have been expected at the beginning for some of the places.

we can see a higher evapotranspiration index in places where there is
plenty precipitation and evapotranspiration rate by canopy. Hainich and
Hyytiala are represented as the higher, this can be due to their higher
latitudes providing this ecosystems with more precipitation rate.
Guyaflux is at lower latitude and thus more close to the equator. This
means a more tropical weather where high amount of rain and sun is
normally occurring. Collelongo is place in a drier climate with high
radiation but less annual precipitation rates.

\end{document}
